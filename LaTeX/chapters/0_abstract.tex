The calibration of the Heston stochastic volatility model constitutes a non-trivial inverse problem, typically constrained by the computational intensity of standard numerical pricing methods. This study evaluates the longitudinal robustness and accuracy of a \ac{ddn} employed as a surrogate pricing engine within a hybrid calibration framework. By utilizing Sobolev training on a synthetic dataset generated via \ac{lhs}, the \ac{ddn} learns to approximate both option prices and their partial derivatives, thereby facilitating rapid convergence within a classical L-BFGS-B optimization routine. A comprehensive empirical backtest performed on \ac{aapl} options from 2016 to 2023 validates the methodology across diverse market regimes. The framework demonstrates high fidelity, achieving an average out-of-sample \ac{mre} of 4.49\%, closely tracking the in-sample \ac{mre} of 4.29\% and indicating strong generalization capabilities. While calibration error exhibits a positive correlation with realized market volatility, notably during the COVID-19 market crash, the model maintains stability and demonstrates rapid error mean reversion. Furthermore, the methodology proves resilient across varying macroeconomic conditions, including regimes characterized by negative market-implied risk-free rates. These findings confirm that DDN-based calibration effectively resolves the trade-off between computational efficiency and pricing accuracy, offering a scalable solution for real-time risk management.

\textit{Keywords:} Heston Model, Deep Differential Networks, Sobolev Training, Stochastic Volatility, Model Calibration, Deep Learning, Quantitative Finance

\textit{JEL Classification:} G12, C45