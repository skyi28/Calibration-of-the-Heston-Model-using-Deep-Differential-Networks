The results of this study demonstrate that a \ac{ddn}, when utilized as a surrogate pricing engine within a classical optimization framework, constitutes a robust and accurate methodology for calibrating the Heston stochastic volatility model. The findings provide definitive answers to the primary research questions posed and offer significant insights into the practical application of deep learning in quantitative finance.

In response to the question of longitudinal robustness, the methodology exhibits stability across the seven-year backtest period. The quantitative analysis reveals a direct and financially intuitive positive correlation between calibration error and realized market volatility. During periods of acute market stress, such as the COVID-19 crash, the validation \ac{mre} remained bounded, with a regime average of 5.33\%, before consistently reverting to its baseline as market conditions stabilized. This resilience confirms the framework's ability to handle extreme market dynamics without catastrophic failure.

Regarding the question of generalization on real-world data, the \ac{ddn} calibration generalizes with high fidelity. The consistently small gap between the calibration and validation \ac{mre}s, as shown in Figure \ref{fig:diagnostic_plots}, indicates that the model does not overfit to the daily calibration set. This is a crucial result, demonstrating that the methodology learns a robust representation of the underlying pricing surface each day, even when fitting to a large cross-section of options averaging 261 unique contracts. Furthermore, the final validation against an analytical pricer provides the most rigorous evidence of the model's quality. The minimal average difference of 0.57 percentage points between the \ac{ddn} surrogate's \ac{mre} (4.33\%) and the true Heston model's \ac{mre} (4.90\%) confirms that the \ac{ddn} is a high-fidelity emulator. The astonishingly small difference of 0.05 percentage points during the "COVID Crash" demonstrates that the \ac{ddn}'s approximation error does not increase even under the most complex market conditions.

These findings significantly validate and extend the foundational work of \textcite{zhang2025calibratinghestonmodeldeep}. While our overall calibration \ac{mre} of 4.33\% appears numerically similar to the 4.64\% reported in the reference study for a 100-option dataset, our result was achieved on a far more complex and diverse calibration surface that was, on average, 2.6 times larger. This suggests that our methodological enhancements, most likely the dynamic calculation of market-implied rates and the use of a quasi-Newton optimizer (L-BFGS-B), have led to superior scalability and robustness under real-world conditions.

The success of the framework can be attributed to the synergistic combination of its core components. The practice of Sobolev training, which penalizes errors in both price and parameter sensitivities, is the key mechanism that enforces the geometric integrity of the pricing surface, enabling the strong generalization observed. This high-fidelity surrogate model, trained with a state-of-the-art `swish` activation function and an AdamW optimizer, is sufficiently precise to guide the final L-BFGS-B optimizer to a high-quality minimum. The final validation with QuantLib's \texttt{AnalyticHestonEngine} confirms that the parameters found are not artifacts of the surrogate model but are genuinely representative of the Heston model's best fit to the market.

From a practical standpoint, this hybrid approach offers a compelling solution to the trade-off between calibration speed and accuracy. The sub-second calibration time makes the methodology suitable for applications requiring frequent recalibration, such as intra-day risk management. The overall true Heston \ac{mre} of 4.90\% should not be viewed as a failure of the calibration, but rather as a realistic measurement of the Heston model's inherent limitations in capturing every feature of the true implied pricing surface. This level of accuracy is highly suitable for applications such as relative value trading, where the goal is to identify significant mispricings against a reliable baseline, and for large-scale risk reporting, where the immediate availability of calibrated Greeks is paramount.

However, the analysis also revealed limitations that suggest avenues for future research. The error analysis in Figure \ref{fig:mre_heatmaps} showed significantly higher relative errors for cheap, out-of-the-money options, a known artifact of the \ac{mre} metric that is amplified by error propagation through the put-call parity formula. Future iterations could address this by training the model directly on the implied volatility surface and implementing a Vega-weighted loss function, which would prioritize the fitting of contracts that are most sensitive to volatility and most relevant to traders and risk managers. Furthermore, while the \ac{ddn} provides a robust point estimate of the Heston parameters, replacing the deterministic network with a Bayesian Neural Network or a deep ensemble could allow the model to provide a credible interval for its predictions. This would offer a quantitative measure of model uncertainty, a feature of significant value for sophisticated risk management applications. Finally, the surrogate modeling approach is particularly promising for models that lack semi-analytical pricing solutions, such as those incorporating jumps or rough volatility, where a network trained on Monte Carlo data could create a fast and accurate pricer for otherwise intractable models. A complementary and highly pragmatic direction for future research is to address the Heston model's residual error with a hybrid, two-stage approach. After the \ac{ddn} calculates the base Heston price, a second "corrector" machine learning model (e.g., a gradient boosting machine) could be trained to predict the systematic pricing error (\(P_{\text{market}} - P_{\text{model}}\)) using option features and the initial price as inputs. The final prediction would then be the sum of the Heston price and this learned correction. This method leverages the \ac{ddn} to enforce the robust, theory-based structure of the Heston model, while the corrector model uses a data-driven approach to learn the model's known systematic biases, such as the volatility smirk. Such a "theory-plus-correction" framework could significantly reduce the final \ac{mre} and push the methodology's accuracy into a range suitable for more demanding pricing applications.