\documentclass[12pt,a4paper]{article}

% Packages
\usepackage[utf8]{inputenc}
\usepackage[T1]{fontenc}
\usepackage{lmodern}
\usepackage[english]{babel}
\usepackage{amsmath,amssymb}
\usepackage{graphicx}
\usepackage{hyperref}
\usepackage{float}
\usepackage{booktabs}
\usepackage[a4paper, left=3cm, right=3cm, top=2cm, bottom=2cm]{geometry}
\usepackage{threeparttable}
\usepackage{makecell}
\usepackage{multirow}
\usepackage{fancyhdr}
\usepackage{listings}
\usepackage{xcolor}
\usepackage{titlesec}
\usepackage{csquotes}
\usepackage{caption}
\usepackage{pgffor}
\usepackage{acro}
\usepackage{tikz}
\usetikzlibrary{
    shapes.geometric,
    positioning,
    arrows.meta,
    calc,
    fit,
}

\usepackage[style=apa, backend=biber]{biblatex}
\addbibresource{references.bib}

% Page layout
\setlength{\parskip}{1em}
\setlength{\parindent}{0em}

% Header and Footer settings
\pagestyle{fancy}
\fancyhf{}
\rhead{\thepage}
\lhead{Calibration of the Heston Model using Neural Networks}

% Section title spacing
\titlespacing*{\section}
  {0pt}      % left indent
  {2pt}      % space before
  {0pt}      % space after

\titlespacing*{\subsection}
  {0pt}
  {2pt}
  {0pt}

% Listings settings
\lstset{
  basicstyle=\ttfamily\small,
  backgroundcolor=\color{gray!10},
  frame=single,
  breaklines=true,
  captionpos=b,
  language=Python
}

% Acroynms
\DeclareAcronym{aapl}{
    short={AAPL},
    long={Apple Inc.}
}
\DeclareAcronym{ai}{
	short={AI},
	long={Artificial Intelligence}
}
\DeclareAcronym{atm}{
	short={ATM},
	long={At-the-Money}
}
\DeclareAcronym{ddn}{
    short={DDN},
    long={Deep Differential Network}
}
\DeclareAcronym{itm}{
    short={ITM},
    long={In-the-Money}
}
\DeclareAcronym{lhs}{
    short={LHS},
    long={Latin Hypercube Sampling}
}
\DeclareAcronym{mse}{
    short={MSE},
    long={Mean Squared Error}
}
\DeclareAcronym{mre}{
    short={MRE},
    long={Mean Relative Error}
}
\DeclareAcronym{otm}{
    short={OTM},
    long={Out-of-the-Money}
}

% Actual Content
\begin{document}

% Title Page
\begin{titlepage}
	\centering
	\includegraphics[width=0.4\textwidth]{images/zhaw_logo.png}
	\vfill
	{\Huge\bfseries Calibration of the Heston Model using Neural Networks\par}
	\vspace{1.0cm}
	{\Large Machine Learning and Deep Learning\par}
	\vspace{1.0cm}
	\vfill
	\begin{minipage}{0.8\textwidth}
		\large
		\begin{tabular}{@{}ll}
			\textbf{Authors:}       & B. Grimus \& A. Mosleh-Tehrani \\
			\textbf{Study Program:} & Banking \& Finance             \\
			\textbf{Supervisor:}    & Dr. Fazlija Bledar             \\
		\end{tabular}
	\end{minipage}
	\vfill
	{\large
		ZHAW School of Management and Law\par
		Submitted on: 01. December 2025\par
	}
\end{titlepage}
\clearpage

% Abstract
\pagenumbering{gobble}
\section*{Abstract}
The calibration of the Heston stochastic volatility model constitutes a non-trivial inverse problem, typically constrained by the computational intensity of standard numerical pricing methods. This study evaluates the longitudinal robustness and accuracy of a \ac{ddn} employed as a surrogate pricing engine within a hybrid calibration framework. The \ac{ddn} is trained exclusively on a synthetic dataset generated via \ac{lhs} using Sobolev training, ensuring no prior exposure to historical market data. The network approximates both option prices and their partial derivatives, thereby facilitating rapid convergence within a classical L-BFGS-B optimization routine. 

To validate the methodology, a comprehensive empirical backtest was performed on \ac{aapl} options from 2016 to 2023. For each trading day, the option chain was partitioned into a calibration set (80\%) for parameter inference and a distinct hold-out validation set (20\%) for pricing assessment. The framework demonstrates high fidelity, achieving an average \ac{mre} of 4.34\% on the hold-out validation set. This metric closely tracks the calibration set \ac{mre} of 4.14\%, indicating that the inferred parameters generalize well to unseen contracts despite the \ac{ddn} relying solely on synthetic pre-training. While calibration error exhibits a positive correlation with realized market volatility, notably during the COVID-19 market crash, the model maintains stability and demonstrates rapid error mean reversion. Furthermore, the methodology proves resilient across varying macroeconomic conditions, including regimes characterized by negative market-implied risk-free rates. These findings confirm that \ac{ddn}-based calibration effectively resolves the trade-off between computational efficiency and pricing accuracy, offering a scalable solution for real-time risk management.

\textit{Keywords:} Heston Model, Deep Differential Networks, Sobolev Training, Stochastic Volatility, Model Calibration, Deep Learning, Quantitative Finance

\textit{JEL Classification:} G12, C45
\clearpage

\pagenumbering{roman}
\tableofcontents
\clearpage
\newpage

% Abbreviations
\printacronyms
% \newpage

% List of Figures
\listoffigures
% \newpage

% List of Tables
\listoftables
\newpage

% Switch page numbering from roman to arabic
\pagenumbering{arabic}

% Chapters
\section{Introduction}
\label{sec:introduction}
The Heston stochastic volatility model remains a cornerstone of modern quantitative finance, offering a more realistic framework for pricing European options than its constant-volatility predecessors by capturing empirical features such as volatility clustering and the volatility smile \parencite{heston1993}. The calibration of the Heston model, the process of inferring its unobservable parameters from market option prices, is a critical task for risk management and the pricing of exotic derivatives. This process, however, constitutes a non-trivial inverse problem characterized by a high-dimensional, non-convex error surface. Traditional calibration methods, which rely on repeatedly evaluating the model's semi-analytical pricing formula within an optimization loop, are computationally intensive and sensitive to the choice of initial parameters \parencite{escobar2016parametersrecoverycalibrationhestonmodel}, making them challenging to deploy in settings that require real-time recalibration.

In recent years, deep learning has emerged as a promising alternative to address the dual challenges of computational efficiency and calibration accuracy. The literature has explored several distinct approaches. One prominent method involves training a neural network as a surrogate model to approximate the complex mapping from model parameters and option characteristics to a final option price. This replaces the slow numerical pricing engine with a fast neural network inference \parencite{liu2019neuralnetworkbasedframeworkfinancialmodelcalibration}. A related technique employs a two-stage hybrid framework, where one network approximates the market price surface and a second network learns to correct the systematic residual errors of a traditionally calibrated Heston model \parencite{zadgar2025deeplearningenhancedcalibrationheston}. A third paradigm formulates calibration as an inverse mapping problem, training a network to learn the direct mapping from market observables, such as an asset's historical time series, to the underlying Heston parameters \parencite{leite2021deeponetsfinanceapproachcalibratehestonmodel}. While innovative, this latter approach addresses the problem of parameter estimation from historical paths rather than the industry-standard problem of calibration to a cross-section of current market option prices.

An advancement in this field is the development of \ac{ddn}, which are trained using a Sobolev-style \parencite{czarnecki2017sobolevtrainingneuralnetworks} loss function to learn not only the option price but also its partial derivatives with respect to the model parameters \parencite{zhang2025calibratinghestonmodeldeep}. By embedding this structural information, \ac{ddn}s can serve as highly accurate pricing engines for fast, gradient-based calibration routines. However, the performance of these advanced frameworks, including \ac{ddn}s, has primarily been evaluated on static datasets or under controlled conditions. A comprehensive validation of their longitudinal robustness, generalization capabilities on large and complex real-world option surfaces, and stability across diverse market and macroeconomic regimes remains an open area of investigation.

This study aims to fill this gap by conducting a rigorous, multi-year historical backtest of a \ac{ddn}-based calibration methodology. We employ a \ac{ddn} not as a standalone calibrator, but as a high-speed, high-fidelity surrogate pricing engine within a classical quasi-Newton optimization framework (L-BFGS-B, \textcite{byrd1995limitedmemoryalgorithmboundconstrainedoptimization}). This hybrid approach is systematically evaluated through a comprehensive longitudinal backtest on a large dataset of \ac{aapl} option prices from 2016 to 2023. The framework incorporates practical adaptations for real-world data, including a dynamic, daily estimation of the market-implied risk-free rate derived from put-call parity.

Specifically, this research seeks to answer several key questions. First, how robust is the calibration accuracy of the \ac{ddn}-based Heston model when applied longitudinally over a multi-year period that encompasses diverse market regimes, including periods of low volatility, market crashes, and changing interest rate environments? Second, to what extent does the methodology generalize from the in-sample (calibration) set to an out-of-sample (test) set on a daily basis, and is there evidence of significant overfitting when fitting to a large cross-section of options? Finally, what is the quantitative relationship between calibration error and realized market volatility, and does the methodology's accuracy degrade or become unstable during periods of extreme market stress?
% \newpage

\section{Methodology}
\label{sec:methodology}
This section outlines the theoretical foundations, computational framework, and empirical validation procedures underpinning this study, following the approach of \textcite{zhang2025calibratinghestonmodeldeep}. The methodology begins with a description of the experimental environment and the underlying Heston model, continues with the generation of synthetic data and the architecture of the \ac{ddn}, and concludes with the design of the historical backtesting protocol.

\subsection{Technical Setup}
All numerical experiments, including data generation, model training, and backtesting, were conducted on a commercially available laptop computer. The system specifications are as follows: an AMD Ryzen 7 7840HS CPU with 16 cores operating at a base frequency of 3.8GHz, 32GB of DDR5 RAM, and an NVIDIA GeForce RTX 4070 Laptop GPU with 8GB of VRAM. The software environment consisted of Windows 11 operating a Windows Subsystem for Linux instance running Ubuntu 24.04.1 LTS. The deep learning models were trained on the GPU, leveraging NVIDIA's CUDA Toolkit version 12.6.

To ensure the determinism and reproducibility of the results, a global random seed of 42 was consistently applied across all relevant software libraries, including \texttt{NumPy}, \texttt{TensorFlow}, and Python's native \texttt{random} module. This practice guarantees that the processes of synthetic data generation, \ac{ddn} weight initialization, and data partitioning remain identical across multiple executions.

\subsection{The Heston Stochastic Volatility Model}
The Heston model, introduced by \textcite{heston1993}, is a stochastic volatility model that posits the variance of an underlying asset is not constant but follows its own random process.

Under the risk-neutral measure $\mathbb{Q}$, the dynamics of the asset price, $S_t$, and its instantaneous variance, $v_t$, are described by a system of two correlated stochastic differential equations:
\begin{equation}
    dS_t = (r_t - q) S_t dt + \sqrt{v_t} S_t dW_t^S
\end{equation}
\begin{equation}
    dv_t = \kappa (\lambda - v_t) dt + \sigma \sqrt{v_t} dW_t^v
\end{equation}
where the two standard Wiener processes, $W_t^S$ and $W_t^v$, have a constant correlation $\rho$, such that $E[dW_t^S dW_t^v] = \rho dt$.

\begin{table}[H]
    \centering
    \begin{threeparttable}
    \caption{Notation for the Heston Model Stochastic Differential Equations.}
    \begin{tabular}{ll}
        \toprule
        Symbol & Description \\
        \midrule
        $S_t$ & Price of the underlying asset at time $t$. \\
        $v_t$ & Instantaneous variance of the asset price at time $t$. \\
        $r_t$ & The continuously compounded risk-free interest rate (deterministic). \\
        $q$ & The implied dividend yield of the underlying asset. \\
        $\kappa$ & The rate of mean reversion of the variance process. \\
        $\lambda$ & The long-run average variance. \\
        $\sigma$ & The volatility of the variance process (volatility of volatility). \\
        $\rho$ & The correlation coefficient between the two Wiener processes. \\
        $W_t^S, W_t^v$ & Standard Wiener processes under the risk-neutral measure. \\
        \bottomrule
    \end{tabular}
    \end{threeparttable}
\end{table}

The Heston model admits a semi-analytical solution for the price of a European call option, which can be computed by leveraging the inverse Fourier transform of the model's characteristic function. This approach avoids the need for computationally intensive Monte Carlo simulations. The price of a European call option, $C$, is a function of the initial state and model parameters, denoted as $C(S_0, K, r, \tau; \theta_H)$, where $\theta_H = \{\kappa, \lambda, \sigma, \rho, v_0\}$ is the set of unobservable Heston parameters. The pricing formula is expressed in a form analogous to the Black-Scholes model:
\begin{equation}
    C(S_0, K, r, q, \tau; \theta_H) = S_0 e^{-q\tau} \Pi_1 - K e^{-r\tau} \Pi_2
\end{equation}
The terms $\Pi_1$ and $\Pi_2$ represent risk-neutral probabilities. In the Heston framework, they are computed via numerical integration of the characteristic function of the log-asset price:
\begin{equation}
    \Pi_1 = \frac{1}{2} + \frac{1}{\pi} \int_0^\infty \text{Re}\left[ \frac{e^{-iu k} \phi_\tau(u-i)}{iu} \right] du
\end{equation}
\begin{equation}
    \Pi_2 = \frac{1}{2} + \frac{1}{\pi} \int_0^\infty \text{Re}\left[ \frac{e^{-iu k} \phi_\tau(u)}{iu} \right] du
\end{equation}
where $\phi_\tau(u)$ is the characteristic function of the logarithm of the asset price at maturity, incorporating the risk-neutral drift $(r - q)$, $k = \ln(K)$, and $i$ is the imaginary unit.

\begin{table}[H]
    \centering
    \begin{threeparttable}
    \caption{Notation for the Heston Semi-Analytical Pricing Formula.}
    \begin{tabular}{ll}
        \toprule
        Symbol & Description \\
        \midrule
        $C$ & Price of the European call option. \\
        $S_0$ & Initial price of the underlying asset. \\
        $K$ & Strike price of the option. \\
        $r$ & The continuously compounded risk-free interest rate. \\
        $q$ & The continuous dividend yield. \\
        $\tau$ & Time to maturity of the option, in years. \\
        $\theta_H$ & The set of Heston parameters $\{\kappa, \lambda, \sigma, \rho, v_0\}$. \\
        $\Pi_1, \Pi_2$ & Risk-neutral probabilities derived from the characteristic function. \\
        $k$ & Log-strike, defined as the natural logarithm of the strike price, $\ln(K)$. \\
        $\phi_\tau(u)$ & The characteristic function of the log-asset price at maturity $\tau$. \\
        $u$ & The integration variable. \\
        $i$ & The imaginary unit, satisfying $i^2 = -1$. \\
        $\text{Re}[z]$ & A function that returns the real part of a complex number $z$. \\
        \bottomrule
    \end{tabular}
    \end{threeparttable}
\end{table}

\subsection{Synthetic Data Generation}
A large-scale synthetic dataset was generated to serve as the training corpus for the \ac{ddn}. This dataset is designed to approximate the Heston pricing function over a wide and diverse parameter space.

To ensure an efficient and uniform coverage of the high-dimensional parameter space, \ac{lhs} introduced by \textcite{mckay1979latinhypercubesampling} was employed. This quasi-random sampling technique divides each parameter's domain into equally probable intervals, ensuring that samples are drawn from all regions of the input space. The domains for the Heston parameters and market variables were defined as follows: $\kappa \in [0.01, 5.0]$, $\lambda \in [0.0, 1.0]$, $\sigma \in [0.1, 1.0]$, $\rho \in [-0.99, 0.0]$, $v_0 \in [0.01, 1.0]$, $r \in [-0.03, 0.1]$, $q \in [0.00, 0.05]$, $\tau \in [5/365, 2.5]$, and log-moneyness $\ln(K/S_0) \in [-1.0, 1.0]$. A total of 200,000 unique parameter vectors were generated.

For each generated parameter vector, the corresponding ground-truth European call option price was computed using the \texttt{AnalyticHestonEngine} from the open-source \texttt{QuantLib} financial library. Subsequently, the first-order partial derivatives of the option price with respect to each of the five Heston parameters ($\partial C / \partial\kappa, \partial C / \partial\lambda, \dots, \partial C / \partial v_0$) were numerically approximated using a central finite difference scheme. The final dataset consists of 200,000 samples, each containing an 9-dimensional feature vector and a 6-dimensional label vector (one price and five gradients).

\subsection{Descriptive Data Analysis}
\label{subsec:descriptive_data_analysis}
A descriptive statistical analysis was performed on both the generated synthetic dataset and the historical \ac{aapl} options data to understand their underlying distributions and characteristics. This analysis informs the data filtering protocol and helps validate the representativeness of the synthetic dataset.

The statistics for the synthetic dataset, detailed in Tables \ref{tab:synthetic_inputs} and \ref{tab:synthetic_outputs}, reflect the properties of the \ac{lhs} method. For all input parameters, the mean is closely aligned with the median (50th percentile), skewness is approximately zero, and kurtosis is approximately -1.20, which is characteristic of a uniform distribution. This confirms that the sampling strategy successfully generated a well-distributed and unbiased representation of the parameter space. In contrast, the output labels, particularly the parameter gradients, exhibit significant positive skewness and high kurtosis (leptokurtosis). For instance, the gradient with respect to kappa ($\text{d}\_\kappa$) shows a skewness of 4.76 and a kurtosis of 70.81. This indicates that the sensitivities of the Heston model are not uniformly distributed and possess fat tails with extreme outliers, a critical feature for the \ac{ddn} to learn.

\begin{table}[H]
    \centering
    \caption{Descriptive Statistics for Synthetic Dataset Input Parameters.}
    \label{tab:synthetic_inputs}
    \begin{threeparttable}
    \begin{tabular}{lrrrrr}
        \toprule
        Statistic & $\kappa$ & $\lambda$ & $\sigma$ & $\rho$ & $v_0$ \\
        \midrule
        Mean & 2.50 & 0.50 & 0.55 & -0.49 & 0.51 \\
        Std. Dev. & 1.44 & 0.29 & 0.26 & 0.29 & 0.29 \\
        Min & 0.01 & 0.00 & 0.10 & -0.99 & 0.01 \\
        25\% & 1.26 & 0.25 & 0.32 & -0.74 & 0.26 \\
        50\% & 2.50 & 0.50 & 0.55 & -0.49 & 0.51 \\
        75\% & 3.75 & 0.75 & 0.77 & -0.25 & 0.75 \\
        Max & 5.00 & 1.00 & 1.00 & -0.00 & 1.00 \\
        Skewness & 0.00 & -0.00 & 0.00 & -0.00 & -0.00 \\
        Kurtosis & -1.20 & -1.20 & -1.20 & -1.20 & -1.20 \\
        \midrule
        \textbf{Statistic} & $r$ & $q$ & $\tau$ & $\log(K/S_0)$ \\
        \midrule
        Mean & 0.04 & 0.03 & 1.26 & -0.00 \\
        Std. Dev. & 0.04 & 0.01 & 0.72 & 0.58 \\
        Min & -0.03 & 0.00 & 0.01 & -1.00 \\
        25\% & 0.00 & 0.01 & 0.64 & -0.50 \\
        50\% & 0.04 & 0.03 & 1.26 & -0.00 \\
        75\% & 0.07 & 0.04 & 1.88 & 0.50 \\
        Max & 0.10 & 0.05 & 2.50 & 1.00 \\
        Skewness & -0.00 & -0.00 & -0.00 & 0.00 \\
        Kurtosis & -1.20 & -1.20 & -1.20 & -1.20 \\
        \bottomrule
    \end{tabular}
    \begin{tablenotes}
        \item Note: All values are rounded to two decimal places.
    \end{tablenotes}
    \end{threeparttable}
\end{table}

\begin{table}[H]
    \centering
    \caption{Descriptive Statistics for Synthetic Dataset Output Labels (Price and Gradients).}
    \label{tab:synthetic_outputs}
    \begin{threeparttable}
    \begin{tabular}{lrrrrrr}
        \toprule
        Statistic & Price & $\text{d}\_\kappa$ & $\text{d}\_\lambda$ & $\text{d}\_\sigma$ & $\text{d}\_\rho$ & $\text{d}\_{v_0}$ \\
        \midrule
        Mean & 0.30 & 0.00 & 0.13 & -0.01 & 0.01 & 0.07 \\
        Std. Dev. & 0.20 & 0.03 & 0.12 & 0.02 & 0.02 & 0.07 \\
        Min & 0.00 & -0.30 & -0.00 & -0.32 & -0.02 & -0.00 \\
        25\% & 0.11 & -0.00 & 0.03 & -0.02 & -0.00 & 0.03 \\
        50\% & 0.29 & 0.00 & 0.11 & -0.01 & 0.00 & 0.06 \\
        75\% & 0.48 & 0.01 & 0.20 & 0.00 & 0.02 & 0.10 \\
        Max & 0.78 & 0.75 & 2.83 & 0.04 & 0.32 & 1.09 \\
        Skewness & 0.10 & 4.76 & 1.72 & -2.88 & 2.66 & 2.47 \\
        Kurtosis & -1.27 & 70.81 & 7.31 & 13.57 & 11.31 & 12.23 \\
        \bottomrule
    \end{tabular}
    \begin{tablenotes}
        \item Note: All values are rounded to two decimal places.
    \end{tablenotes}
    \end{threeparttable}
\end{table}

A correlation analysis was performed on the synthetic dataset to understand the linear relationships engineered by the Heston model, as depicted in Figure \ref{fig:synthetic_correlation}.

\begin{figure}[H]
    \centering
    \includegraphics[width=1\textwidth]{../data/descriptive_analysis/synthetic_data/correlation_heatmap.png}
    \caption{Correlation matrix of the input parameters, option price, and parameter gradients in the synthetic Heston dataset. The near-zero correlations among the input parameters confirm the effectiveness of the \ac{lhs} method.}
    \label{fig:synthetic_correlation}
\end{figure}

The heatmap reveals several key structural properties of the dataset that are critical for the training process:

\textit{Input Parameter Independence:} A defining feature is the block of near-zero correlations among all input parameters in the upper-left quadrant of the matrix. This is a direct and desirable consequence of the \ac{lhs} method, which is designed to generate input vectors that are orthogonal, ensuring that the \ac{ddn} can learn the effect of each parameter independently without multicollinearity issues.

\textit{Primary Price Drivers:} The option price is most strongly influenced by moneyness, with a correlation of -0.90. This confirms the fundamental principle that a call option's value decreases as it becomes more \ac{otm}. The next most significant factor is the time to maturity (tau), with a positive correlation of 0.27, reflecting the option's time value. The variance parameters (lambda and v0) exhibit weaker positive linear relationships with the price.

\textit{Gradient Interdependencies:} The most complex relationships are observed among the output gradients (the lower-right quadrant). There are strong correlations between several gradients, such as the negative correlation of -0.73 between the sensitivity to vol-of-vol (d\_sigma) and the sensitivity to correlation (d\_rho). Furthermore, the gradients are highly correlated with the input variables. For instance, the sensitivity to long-run variance (d\_lambda) has a strong positive correlation of 0.64 with tau, indicating that the model's sensitivity to this parameter increases with the option's maturity.

This analysis demonstrates that while the inputs to the model are independent by design, the outputs (the price and its sensitivities) form a highly structured and interdependent surface.

The historical dataset of \ac{aapl} options underwent a rigorous filtering process to isolate a high-quality subset for calibration. This multi-stage procedure is designed to mitigate the influence of market microstructure noise and to focus the calibration on the most liquid and informative contracts. Table \ref{tab:data_attrition} presents the data attrition at each stage of this process. 

The initial dataset contained over 1.5 million records. The first step removes options with a mid-price below \$0.50. This is done to exclude illiquid "penny options", whose prices are often characterized by wide relative bid-ask spreads and pricing noise \parencite{figlewski2008estimatingtheimpliedriskneutraldensityusmarketportfolio}, making them unreliable for model fitting. Next, options with fewer than five days to maturity are excluded since they may induce liquidity-related biases \parencite{bakshi1997empiricalperformancealternativeoptionpricingmodels}. Finally, the dataset is restricted to options within a log-moneyness range of [-0.25, 0.25]. As shown in the table, this is the most restrictive step, retaining only near-the-money contracts. The rationale for this is twofold: first, this region contains the highest trading volume and liquidity, providing the most reliable market prices \parencite{bakshi1997empiricalperformancealternativeoptionpricingmodels}. Second, near-the-money options are the most sensitive to changes in volatility (i.e., they have the highest vega) \parencite[pp. 416-417]{hull2015optionsfutures} and thus contain the most relevant information for calibrating the parameters of a stochastic volatility model. 

This systematic reduction, resulting in a final count of 643,953 options, confirms that the subsequent analysis and calibration are concentrated on the most robust and actively traded segment of the options market.

\begin{table}[H]
    \centering
    \caption{Data Filtering Process and Attrition for Historical \ac{aapl} Options.}
    \label{tab:data_attrition}
    \begin{threeparttable}
    \begin{tabular}{llrrr}
        \toprule
        Step & Description & Remaining & Removed & Remaining \% \\
        \midrule
        1 & Raw Data Extraction & 1,562,105 & 0 & 100.00 \\
        2 & Price Filter ($>$\$0.50) & 1,236,446 & 325,659 & 79.15 \\
        3 & Maturity Filter ($>$5 days) & 1,234,300 & 2,146 & 79.02 \\
        4 & Moneyness Filter & 643,953 & 590,347 & 41.22 \\
        \bottomrule
    \end{tabular}
    \end{threeparttable}
\end{table}

A preliminary cleaning step was performed on all column headers to remove extraneous whitespace and special characters, especially the square brackets. The following definitions were then used throughout the descriptive analysis and backtesting procedures:

Underlying Asset Price ($S_0$): This was directly mapped from the \texttt{UNDERLYING\_LAST} column in the historical dataset, representing the last traded price of the underlying asset for a given option quote.
    
Strike Price ($K$): This was directly mapped from the \texttt{STRIKE} column.
    
Market Option Price ($C_{\text{market}}$): The target price for calibration was defined as the midpoint of the bid and ask prices. This was calculated as $(\texttt{C\_BID} + \texttt{C\_ASK}) / 2.0$. This standard practice helps to mitigate the effects of bid-ask bounce and provide a more stable price reference.
    
Time to Maturity ($\tau$): The model requires time to maturity in annualized units. This was derived from the \texttt{DTE} (Days to Expiration) column by dividing its value by 365.0.
    
Log-Moneyness ($\ln(K/S_0)$): The \ac{ddn} was trained using log-moneyness as a feature to leverage the homogeneity property of the pricing model. This input was not taken directly from the data but was computed as the natural logarithm of the ratio of the newly defined Strike Price ($K$) and Underlying Asset Price ($S_0$).

The descriptive statistics for the final, filtered historical dataset are provided in Tables \ref{tab:hist_stats1} and \ref{tab:hist_stats2}. In stark contrast to the synthetic data, the historical market data exhibits significant non-uniformity. The implied volatility for calls ($\text{C}\_\text{IV}$), for example, displays extreme right skewness (6.84) and exceptionally high kurtosis (103.81). This is characteristic of financial market data and reflects the presence of volatility spikes and tail events, such as market crashes. The distribution of time to maturity ($\text{Tau}\_\text{Years}$) is also heavily right-skewed (1.65), indicating a higher concentration of shorter-dated options in the dataset. These properties underscore the challenging nature of calibrating models to real-world market conditions.

\begin{table}[H]
    \centering
    \caption{Descriptive Statistics for Filtered Historical \ac{aapl} Options (Part 1).}
    \label{tab:hist_stats1}
    \begin{threeparttable}
    \begin{tabular}{lrrrrr}
        \toprule
        Statistic & Underlying & Strike & Call Price & DTE & Tau (Years) \\
        \midrule
        Mean & 178.82 & 174.85 & 17.51 & 159.07 & 0.44 \\
        Std. Dev. & 74.75 & 78.74 & 15.95 & 210.88 & 0.58 \\
        Min & 90.34 & 75.00 & 0.51 & 0.00 & 0.00 \\
        25\% & 132.26 & 125.00 & 5.40 & 21.04 & 0.06 \\
        50\% & 155.31 & 150.00 & 13.80 & 45.00 & 0.12 \\
        75\% & 197.96 & 195.00 & 25.03 & 217.96 & 0.60 \\
        Max & 506.19 & 645.00 & 163.03 & 898.96 & 2.46 \\
        Skewness & 1.81 & 1.88 & 1.84 & 1.65 & 1.65 \\
        Kurtosis & 3.46 & 4.20 & 5.60 & 1.76 & 1.76 \\
        \bottomrule
    \end{tabular}
    \end{threeparttable}
    \begin{tablenotes}
        \item Note: All values are rounded to two decimal places.
    \end{tablenotes}
\end{table}

\begin{table}[H]
    \centering
    \caption{Descriptive Statistics for Filtered Historical \ac{aapl} Options (Part 2).}
    \label{tab:hist_stats2}
    \begin{threeparttable}
    \begin{tabular}{lrrrrr}
        \toprule
        Statistic & C\_IV & P\_IV & $\log(K/S_0)$ & C\_BID & C\_ASK \\
        \midrule
        Mean & 0.35 & 0.33 & -0.03 & 17.24 & 17.78 \\
        Std. Dev. & 0.18 & 0.15 & 0.12 & 15.73 & 16.17 \\
        Min &  0.00 &  0.00 & -0.25 & 0.00 & 0.50 \\
        25\% & 0.26 & 0.24 & -0.13 & 5.30 & 5.50 \\
        50\% & 0.31 & 0.30 & -0.04 & 13.55 & 14.01 \\
        75\% & 0.38 & 0.37 & 0.05 & 24.70 & 25.39 \\
        Max & 9.89 & 3.38 & 0.25 & 161.10 & 164.96 \\
        Skewness & 6.84 & 3.24 & 0.31 & 1.84 & 1.84 \\
        Kurtosis & 103.81 & 20.29 & -0.70 & 5.65 & 5.55 \\
        \bottomrule
    \end{tabular}
    \begin{tablenotes}
        \item Note: All values are rounded to two decimal places.
    \end{tablenotes}
    \end{threeparttable}
\end{table}

To further investigate the relationships within the filtered historical dataset, a Pearson correlation matrix was computed, as visualized in Figure \ref{fig:correlation_heatmap}. The heatmap reveals several significant relationships that are consistent with established financial theory and market structure.

\begin{figure}[h!]
    \centering
    \includegraphics[width=1\textwidth]{../data/descriptive_analysis/historic_aapl_data/correlation_heatmap.png}
    \caption{Correlation matrix of the key variables in the filtered historical AAPL options dataset. The values indicate the Pearson correlation coefficient, with 1.00 (deep red) representing a perfect positive correlation and -1.00 (deep blue) representing a perfect negative correlation.}
    \label{fig:correlation_heatmap}
\end{figure}

As expected, the call price exhibits a perfect correlation of 1.00 with the bid and ask price for call options. Similarly, `Tau\_Years` and `DTE` are perfectly correlated as one is a direct scaling of the other. More substantive insights can be drawn from the relationships between pricing variables:

\textit{Price-Driving Factors:} The price of a call option demonstrates strong, theoretically consistent correlations with its primary drivers. There is a moderate positive correlation with the underlying price (0.46), reflecting the option's delta. A positive correlation with the maturity (0.32) confirms that options with longer maturities hold more time value. The strong negative correlation with moneyness (-0.62) is particularly important, as it correctly captures that as the strike price increases relative to the spot price, the value of a call option decreases.

\textit{Market Structure:} A very high positive correlation of 0.95 is observed between the underlying price and the strike. This does not imply a direct pricing relationship but is an artifact of the long time-series; as the price of \ac{aapl} stock trended upwards from 2016 to 2023, the range of available strike prices listed on the exchange shifted higher in tandem.

\textit{Volatility Structure:} The implied volatilities for calls (C\_IV) and puts (P\_IV) are highly correlated (0.86), indicating that they are driven by the same underlying market sentiment and uncertainty. Furthermore, the negative correlation of -0.32 between the implied volatility of call options and the log-moneyness provides clear evidence of the well-documented volatility skew in equity markets, where implied volatility tends to decrease for \ac{otm} calls (i.e., as log-moneyness increases).

Overall, the correlation analysis confirms that the variables within the historical dataset exhibit financially sound and predictable relationships, providing a robust basis for the subsequent calibration experiments.

\subsection{Deep Differential Network Architecture and Training}
A \ac{ddn} was constructed to serve as a surrogate for the Heston pricing function and its derivatives. The network architecture consists of an input layer with 8 neurons, corresponding to the five Heston parameters and the three market variables ($r, \tau, \log(K/S_0)$), and a single-neuron output layer that predicts the option price normalized by the underlying asset price, $C/S_0$. The optimal internal topology of the \ac{ddn}, including the number and dimension of hidden layers, was determined through a systematic hyperparameter search conducted using the Hyperband algorithm. The search space explored by this optimization process, which included multiple candidates for the hidden layer activation function, is detailed in Table \ref{tab:hyperparameter_space}. Independently of the hidden layer configuration, the output layer consistently utilizes a softplus activation. This choice is a fixed aspect of the model design, implemented to ensure that all predicted option prices are strictly positive, thereby satisfying a fundamental no-arbitrage condition.

The \ac{ddn} was trained using a differential learning approach, also known as Sobolev training \parencite{czarnecki2017sobolevtrainingneuralnetworks}. This paradigm requires the \ac{ddn} to minimize not only the error in the predicted option price but also the error in its predicted partial derivatives. This is achieved by implementing a custom training step that utilizes a nested gradient tape in \texttt{TensorFlow} to compute both the \ac{ddn} output and its gradients with respect to the inputs. The composite loss function, $L_{\text{total}}$, is a weighted sum of the price loss and the gradient loss:
\begin{equation}
    L_{\text{total}} = \alpha \cdot L_{\text{price}} + L_{\text{gradients}}
\end{equation}
where $\alpha=10.0$ is a weighting factor, $L_{\text{price}}$ is the \ac{mse} between the predicted and true prices, and $L_{\text{gradients}}$ is the \ac{mse} between the predicted and true parameter gradients. The weightening factor $\alpha$ was introduced to balance the contributions of the two loss components, ensuring that the model adequately learns both the pricing function and its sensitivities.

\begin{table}[H]
    \centering
    \caption{Hyperparameter Search Space for the Hyperband Algorithm.}
    \label{tab:hyperparameter_space}
    \begin{threeparttable}
    \begin{tabular}{lll}
        \toprule
        Hyperparameter & Type & Search Space / Values \\
        \midrule
        Number of Hidden Layers & Integer & [4, 8] with step 1 \\
        Neurons per Hidden Layer & Integer & [32, 256] with step 32 \\
        Dropout Rate & Float & [0.0, 0.5] with step 0.05 \\
        Activation Function & Categorical & \{`swish', `tanh', `softplus'\} \\
        Initial Learning Rate & Categorical & \{1e-2, 1e-3, 5e-4, 1e-4\} \\
        First Decay Epochs & Integer & [25, 200] with step 25 \\
        \bottomrule
    \end{tabular}
    \end{threeparttable}
\end{table}

\begin{table}[H]
    \centering
    \begin{threeparttable}
    \caption{Notation for the Composite Loss Function.}
    \begin{tabular}{ll}
        \toprule
        Symbol & Description \\
        \midrule
        $L_{\text{total}}$ & The total composite loss minimized during training. \\
        $\alpha$ & A scalar hyperparameter weighting the contribution of the price loss. \\
        $L_{\text{price}}$ & The \ac{mse} component for the option price. \\
        $L_{\text{gradients}}$ & The \ac{mse} component for the price gradients. \\
        \bottomrule
    \end{tabular}
    \end{threeparttable}
\end{table}

Prior to training, the input features were scaled to the range $[-1, 1]$ and the output labels to $[0, 1]$ using a MinMaxScaler. A critical step in the differential learning process is the adjustment of the target gradients via the chain rule to account for this scaling transformation. The \ac{ddn} was trained using the AdamW optimizer, which implements decoupled weight decay \parencite{loshchilov2019decoupledweightdecayregularization}, in conjunction with a CosineDecayRestarts learning rate schedule introduced by \textcite{loshchilov2017sgdrstochasticgradientdescent}. This combination of techniques promotes stable convergence to a high-precision solution.

\subsection{Empirical Backtesting Framework}
To validate the performance of the trained \ac{ddn} in a realistic setting, a comprehensive historical backtest was conducted on a dataset of call and put options on \ac{aapl} stock, spanning the period from January 2016 to March 2023. On each trading day, the raw data was subjected to the filtering protocol detailed in Section \ref{subsec:descriptive_data_analysis} to isolate a stable subset of liquid options for analysis. The backtest then proceeded on a day-by-day basis, executing the following calibration and evaluation procedure:

\begin{enumerate}
    \item \textit{Interest Rate and Dividend Yield Estimation:} The risk-free interest rate ($r$) was obtained by linearly interpolating the daily Treasury yield curve to match the specific time to maturity ($\tau$) of each option. Subsequently, the continuous dividend yield ($q$) was dynamically estimated for each day by applying the put-call parity theorem to a set of liquid, \ac{atm} options. The median of the calculated yields from this set was used as the daily input. The dividend yield is derived as:
    \begin{equation}
        q = -\frac{1}{\tau} \ln\left(\frac{C_{\text{market}} - P_{\text{market}} + K e^{-r\tau}}{S_0}\right)
    \end{equation}
    This approach explicitly decouples the risk-free rate from the dividend yield, ensuring that the model correctly handles both the drift adjustment and the discounting of payoffs.

    \item \textit{Data Partitioning:} On each trading day, the filtered option chain was independently partitioned into a calibration set (80\% of contracts) and a held-out validation set (20\%) by sampling the options randomly.

    \item \textit{Optimization:} The Heston parameters were calibrated by minimizing the pricing error on the calibration set. The \ac{ddn} served as the surrogate pricing engine within a \texttt{scipy.optimize.minimize} routine, which employed the L-BFGS-B algorithm. The \ac{ddn} supplied both the objective function value and its exact analytical Jacobian (the "Neural Greeks") to the optimizer, enabling rapid convergence.

    \item \textit{Multi-Start Optimization:} To enhance the probability of finding a global optimum and avoid local minima, the optimization process was initiated from three random starting points, with the best-fitting parameter set being retained for that day's evaluation.
\end{enumerate}

The primary metric for evaluating the accuracy of the daily calibration is the \ac{mre}, which measures the average relative deviation between model prices and market prices. This metric was calculated separately for the calibration set (in-sample error) and the held-out test set (out-of-sample error) for each trading day to assess the model's ability to generalize to unseen options. The \ac{mre} is defined as:
\begin{equation}
    MRE = \frac{1}{M} \sum_{m=1}^{M} \frac{\left| C_{\text{model}}^{(m)} - C_{\text{market}}^{(m)} \right|}{C_{\text{market}}^{(m)}}
\end{equation}

\begin{table}[H]
    \centering
    \begin{threeparttable}
    \caption{Notation for the Implied Rate and \ac{mre} Formulas.}
    \begin{tabular}{ll}
        \toprule
        Symbol & Description \\
        \midrule
        $r_{\text{implied}}$ & The risk-free rate. \\
        $P_{\text{market}}$ & The observed market price of a European put option. \\
        $MRE$ & \ac{mre}. \\
        $M$ & The total number of options in the evaluation set. \\
        $C_{\text{model}}^{(m)}$ & The price of the $m$-th option as computed by the \ac{ddn}. \\
        $C_{\text{market}}^{(m)}$ & The observed market price of the $m$-th call option. \\
        \bottomrule
    \end{tabular}
    \end{threeparttable}
\end{table}
% \newpage

\section{Results}
\label{sec:results}
\subsection{Hyperparameter Optimization Results}
To identify the optimal configuration for the \ac{ddn}, a hyperparameter search was conducted using the Hyperband algorithm as described in the methodology. The search aimed to minimize the validation loss on a held-out portion of the synthetic dataset. The resulting optimal set of hyperparameters, which was subsequently used for the final model training and backtesting, is summarized in Table \ref{tab:optimal_hyperparameters}.

This configuration achieved a final validation loss ($L_{\text{total}}$) of approximately $2.09 \times 10^{-5}$ during the tuning process which is comparable to the validation loss found by \textcite{zhang2025calibratinghestonmodeldeep} who reported $3.26 \times 10^{-5}$. A notable outcome of the optimization is the selection of a zero dropout rate, indicating that for this specific architecture and dataset, the combination of the AdamW optimizer's weight decay and the complexity of the differential learning task provided sufficient regularization against overfitting.

\begin{table}[H]
    \centering
    \caption{Optimal Hyperparameters Determined by the Hyperband Algorithm.}
    \label{tab:optimal_hyperparameters}
    \begin{threeparttable}
    \begin{tabular}{ll}
        \toprule
        Hyperparameter & Optimal Value \\
        \midrule
        Number of Hidden Layers & 7 \\
        Neurons per Hidden Layer & 64 \\
        Dropout Rate & 0.0 \\
        Activation Function & swish \\
        Initial Learning Rate & 0.0005 \\
        First Decay Epochs & 50 \\
        \bottomrule
    \end{tabular}
    \end{threeparttable}
\end{table}

\subsection{Backtesting Performance and Robustness Analysis}
The primary objective of the historical backtest is to evaluate the robustness and accuracy of the \ac{ddn} calibration methodology across a wide range of market conditions. This subsection analyzes the daily calibration performance from 2016 to 2023, with a particular focus on periods of market stress.

The time-series evolution of the daily calibration error is presented in Figure \ref{fig:robustness_analysis}. This figure plots both the in-sample \ac{mre}, calculated on the 80\% of options used for calibration, and the out-of-sample \ac{mre}, calculated on the 20\% held-out set. The 20-day realized volatility of the underlying \ac{aapl} stock is overlaid to provide market context.

\begin{figure}[H]
    \centering
    \includegraphics[width=\textwidth]{../plots/plot_1_stress_test.png}
    \caption{Time-series of daily in-sample and out-of-sample \ac{mre} from the Heston model calibration, plotted against the 20-day realized volatility of \ac{aapl} stock. Key market stress events are annotated with vertical dotted lines.}
    \label{fig:robustness_analysis}
\end{figure}

Several key observations can be drawn from this analysis. First, the out-of-sample \ac{mre} consistently tracks the in-sample \ac{mre} very closely throughout the entire seven-year period. This indicates a very small generalization gap, suggesting that the model is not overfitting to the specific subset of options used for calibration each day and successfully captures the underlying volatility surface. Second, the calibration error is strongly positively correlated with the realized market volatility. During periods of low, stable volatility, the \ac{mre} consistently remains below 5\%. Conversely, spikes in calibration error coincide directly with spikes in realized volatility. This is particularly evident during the annotated stress periods, such as the "Volmageddon" event in early 2018 and the "COVID Crash" in March 2020. During the peak of the COVID-19 crisis, the out-of-sample \ac{mre} reached its maximum observed level of approximately 16\%. However, it is critical to note that the error remained bounded and reverted to its baseline level as market volatility subsided, demonstrating the robustness of the calibration procedure.

To quantify these observations, the backtest period was segmented into five distinct macroeconomic and market regimes. The aggregated performance metrics for each regime are presented in Table \ref{tab:regime_performance}. The analysis confirms that the highest average out-of-sample \ac{mre} occurred during the "COVID Crash" regime, at 6.76\%. This period also exhibited the highest standard deviation of error (2.46\%), indicating more erratic day-to-day calibration performance, which is consistent with the extreme market turbulence at the time. In contrast, the model achieved its highest accuracy during the "Inflation" regime of 2022-2023, with an average out-of-sample \ac{mre} of just 4.09\%.

\begin{table}[H]
    \centering
    \caption{Calibration Performance and Parameter Stability Across Market Regimes.}
    \label{tab:regime_performance}
    \begin{threeparttable}
    \begin{tabular}{lrrrrrrr}
        \toprule
        Regime & Avg \ac{mre} & Std \ac{mre} & Std $\kappa$ & Std $\lambda$ & Std $\sigma$ & Std $\rho$ & Std $v_0$ \\
        \midrule
        \makecell[l]{Pre-Volmageddon \\ \textit{\footnotesize (Jan 2015 - Jan 2018)}} & 0.058 & 0.021 & 0.942 & 0.016 & 0.172 & 0.171 & 0.030 \\
        \makecell[l]{Trade War \\ \textit{\footnotesize (Feb 2018 - Jan 2020)}} & 0.059 & 0.022 & 0.889 & 0.028 & 0.182 & 0.174 & 0.038 \\
        \makecell[l]{COVID Crash \\ \textit{\footnotesize (Feb 2020 - May 2020)}} & 0.068 & 0.025 & 0.851 & 0.031 & 0.217 & 0.247 & 0.160 \\
        \makecell[l]{Recovery \\ \textit{\footnotesize (Jun 2020 - Dec 2021)}} & 0.056 & 0.021 & 0.792 & 0.028 & 0.328 & 0.182 & 0.073 \\
        \makecell[l]{Inflation \\ \textit{\footnotesize (Jan 2022 - Dec 2023)}} & 0.041 & 0.014 & 0.698 & 0.020 & 0.156 & 0.146 & 0.039 \\
        \bottomrule
    \end{tabular}
    \begin{tablenotes}
        \item Note: All values are rounded to three decimal places.
    \end{tablenotes}
    \end{threeparttable}
\end{table}

Furthermore, Table \ref{tab:regime_performance} provides insight into the stability of the calibrated Heston parameters. The standard deviation of the initial variance, $v_0$, was an order of magnitude higher during the "COVID Crash" (0.160) than in calmer periods like "Pre-Volmageddon" (0.030). This is a financially intuitive result, as $v_0$ represents the current level of market variance, which was exceptionally volatile during the crash. Similarly, the volatility of volatility, $\sigma$, shows the highest instability during the "Recovery" period, possibly reflecting market uncertainty about the path of the economic recovery. The fact that the calibrated parameters adapt in a theoretically consistent manner provides further validation for the calibration methodology.

A direct comparison of the overall backtesting performance with the results reported in the reference study by \textcite{zhang2025calibratinghestonmodeldeep} provides further context for the model's accuracy. Across the entire seven-year backtest period, the calibration methodology achieved an average in-sample \ac{mre} of 5.33\% and an average out-of-sample \ac{mre} of 5.55\%. The reference paper evaluates its \ac{ddn} method on datasets of 10, 50, and 100 Microsoft call options, reporting \ac{mre}s of 0.67\%, 1.86\%, and 4.64\%, respectively, demonstrating a clear degradation in performance as the complexity of the calibration surface increases with the number of contracts. A crucial factor in interpreting these results is the size and diversity of the option set used for daily calibration. In this study, the daily calibration was performed on a significantly larger set of contracts, with an average of 261 options in the in-sample set and 65 options in the out-of-sample set.

To further diagnose the performance characteristics of the daily calibration, Figure \ref{fig:diagnostic_plots} provides a more granular analysis of the model's generalization capability and its pricing accuracy at the individual option level.

\begin{figure}[H]
    \centering
    \includegraphics[width=\textwidth]{../plots/plot_5_regression.png}
    \caption{Diagnostic scatter plots of the backtesting performance. The left panel shows the daily out-of-sample \ac{mre} versus the in-sample \ac{mre}, illustrating the model's generalization gap. The right panel plots the calibrated model price against the observed market price for a large sample of individual options.}
    \label{fig:diagnostic_plots}
\end{figure}

The left panel of Figure \ref{fig:diagnostic_plots} directly assesses the generalization gap by plotting the out-of-sample \ac{mre} against the in-sample \ac{mre} for each day of the backtest. The data points are tightly clustered around the 45-degree line of perfect generalization, which represents the ideal scenario where the model performs identically on the held-out test data as it does on the data used for calibration. The consistently small vertical distance between the observed points and this line indicates that the model does not suffer from significant overfitting. This provides strong evidence that the \ac{ddn}-based calibration is learning a robust representation of the underlying volatility surface each day, rather than simply memorizing the prices of the specific contracts in the calibration set.

The right panel provides a complementary view by examining the pricing accuracy at the level of individual options, aggregated across multiple days in the backtest. The plot shows a very high concentration of points along the identity line, where the model price is equal to the market price. This demonstrates the model's high fidelity in replicating market prices across a wide range of absolute values, from near-zero to over \$350. A degree of heteroscedasticity is observable, where the variance of the absolute pricing error increases for higher-priced, deep-\ac{itm} options. This is an expected statistical artifact in financial modeling and does not detract from the overall conclusion that the calibrated model consistently produces prices that are in close agreement with market observations.

\begin{figure}[H]
    \centering
    \includegraphics[width=\textwidth]{../plots/plot_3_price_fit.png}
    \caption{Calibrated \ac{ddn}-Heston model fit against observed market prices for \ac{aapl} call options on June 15, 2020 (Spot: \$342.99). Each subplot represents a different option maturity, demonstrating the model's fit across the term structure.}
    \label{fig:price_fit_panel}
\end{figure}

To provide a more granular, qualitative assessment of the calibration performance on a single trading day, Figure~\ref{fig:price_fit_panel} presents the model-implied prices against observed market prices for \ac{aapl} options on June 15, 2020. A high degree of correspondence is observed between the calibrated \ac{ddn}-Heston model prices and the market quotes, particularly for short- to medium-term maturities. The model successfully captures the characteristic convex decay of the option price as a function of the strike price across the entire term structure presented. A minor, yet systematic, deviation is observable for the longest-dated options (823 days), where the model appears to slightly underprice options with strikes ranging from \$150 to \$500. This discrepancy may be attributable to several factors, including the reduced liquidity and wider bid-ask spreads typical of long-term options, or the inherent limitations of the Heston model in capturing the term structure of volatility over multi-year horizons. Nevertheless, the visual evidence presented in the figure corroborates the low \ac{mre} metrics reported, demonstrating the model's capability to produce a consistent and accurate fit across a wide range of strikes and maturities for a given day.

Taken together, these diagnostic plots reinforce the findings from the time-series analysis, confirming that the calibration methodology is not only robust across different market conditions but also demonstrates strong generalization and high pricing fidelity at both the aggregate and individual instrument levels.

\subsection{Error Analysis by Moneyness and Maturity}
To disaggregate the overall performance and identify specific areas of strength and weakness, the out-of-sample \ac{mre} was analyzed across different option moneyness and maturity buckets. Figure \ref{fig:mre_heatmaps} presents this analysis for both call options, which are priced directly by the \ac{ddn}, and put options, whose prices are derived using the put-call parity relationship.

\begin{figure}[H]
    \centering
    \includegraphics[width=\textwidth]{../plots/plot_4_heatmap.png}
    \caption{Out-of-sample \ac{mre} stratified by time to maturity and linear moneyness. The left panel shows the \ac{mre} for call options priced directly by the \ac{ddn}. The right panel shows the \ac{mre} for put options, with prices derived from the \ac{ddn}'s call price predictions via put-call parity.}
    \label{fig:mre_heatmaps}
\end{figure}

The analysis of the call option performance reveals a distinct and theoretically consistent pattern. The model achieves its highest accuracy, with \ac{mre}s as low as 1.4\%, for \ac{itm} (moneyness < 0.95) and \ac{atm} (moneyness 0.95-1.05) contracts, particularly those with longer maturities. This region represents the most liquid and highest-priced segment of the call option market. Conversely, the calibration error is highest for \ac{otm} (moneyness > 1.05) calls, reaching a peak of 15.2\% for contracts with less than three months to expiration. This behavior is a known characteristic of the \ac{mre} metric; \ac{otm} options have very low absolute prices, causing even small absolute pricing errors to translate into large relative errors.

The performance for put options, whose prices are derived via parity, shows a notable asymmetry. The model is accurate for \ac{itm} puts (moneyness > 1.05), where the \ac{mre} is consistently low across all maturities, ranging from 1.2\% to 3.5\%. However, the model exhibits significantly higher errors for \ac{otm} puts (moneyness < 0.95), with the \ac{mre} reaching 26.8\% for short- to medium-dated contracts. This phenomenon is a direct consequence of the error propagation through the put-call parity formula. An \ac{otm} put corresponds to an \ac{itm} call. The model's highly accurate pricing of \ac{itm} calls, when used in the parity equation to price the corresponding (and very cheap) \ac{otm} puts, can result in a large relative error for the put. Conversely, the larger relative errors from cheap \ac{otm} calls do not significantly impact the relative error of the corresponding expensive \ac{itm} puts. Overall, the analysis confirms that the \ac{ddn} calibration is most robust in the \ac{atm} region, which is the most critical for practical risk management and volatility trading applications.

\subsection{Stability Across Interest Rate Regimes}
To assess the impact of the prevailing interest rate and dividend yield environment on calibration accuracy, the relationship between the daily out-of-sample \ac{mre} and the dynamically calculated implied risk-free rate ($r-q$) was examined. The backtesting period from 2016 to 2023 encompassed a wide range of monetary conditions, including periods of near-zero and negative implied rates, providing a robust test of the model's stability. Figure \ref{fig:rate_stability} presents a scatter plot of this relationship for every day in the backtest.

\begin{figure}[H]
    \centering
    \includegraphics[width=1\textwidth]{../plots/plot_2_rate_sensitivity.png}
    \caption{Out-of-sample \ac{mre} as a function of the daily implied risk-free rate ($r-q$). Each point represents a single trading day's calibration result. A linear regression line is overlaid to indicate the general trend.}
    \label{fig:rate_stability}
\end{figure}

The analysis reveals a weak, negative linear relationship between the implied risk-free rate and the calibration error. As the implied rate increases, the out-of-sample \ac{mre} exhibits a modest tendency to decrease. Importantly, there is no evidence of performance degradation or instability in the low or negative implied rate regimes. The calibration errors remain well-distributed and bounded across the entire observed spectrum of rates, from approximately -3\% to +4\%.

A plausible interpretation of this negative trend is that periods of very low or negative implied rates often coincide with periods of market stress, where high dividend yields relative to risk-free rates are more common. As established in the time-series analysis, such periods of market stress are typically associated with higher realized volatility, which is the primary driver of increased calibration error. Therefore, the slightly higher \ac{mre} observed at the lower end of the rate spectrum is likely attributable to the confounding effect of volatility rather than the interest rate level itself. The key finding from this analysis is that the \ac{ddn}-based calibration methodology demonstrates considerable robustness and does not exhibit any systematic failure or bias across the diverse interest rate and dividend yield environments encountered during the seven-year backtest. This validates the effectiveness of the dynamic, parity-based rate estimation procedure.
% \newpage

\section{Discussion}
\label{sec:discussion}
The results of this study demonstrate that a \ac{ddn}, when utilized as a surrogate pricing engine within a classical optimization framework, constitutes a robust and accurate methodology for calibrating the Heston stochastic volatility model. The findings provide definitive answers to the primary research questions posed and offer significant insights into the practical application of deep learning in quantitative finance.

In response to the question of longitudinal robustness, the methodology exhibits stability across the seven-year backtest period. The quantitative analysis reveals a direct and financially intuitive positive correlation between calibration error and realized market volatility. During periods of acute market stress, such as the COVID-19 crash, the validation \ac{mre} remained bounded, with a regime average of 5.33\%, before consistently reverting to its baseline as market conditions stabilized. This resilience confirms the framework's ability to handle extreme market dynamics without catastrophic failure.

Regarding the question of generalization on real-world data, the \ac{ddn} calibration generalizes with high fidelity. The consistently small gap between the calibration and validation \ac{mre}s, as shown in Figure \ref{fig:diagnostic_plots}, indicates that the model does not overfit to the daily calibration set. This is a crucial result, demonstrating that the methodology learns a robust representation of the underlying pricing surface each day, even when fitting to a large cross-section of options averaging 261 unique contracts. Furthermore, the final validation against an analytical pricer provides the most rigorous evidence of the model's quality. The minimal average difference of 0.57 percentage points between the \ac{ddn} surrogate's \ac{mre} (4.33\%) and the true Heston model's \ac{mre} (4.90\%) confirms that the \ac{ddn} is a high-fidelity emulator. The astonishingly small difference of 0.05 percentage points during the "COVID Crash" demonstrates that the \ac{ddn}'s approximation error does not increase even under the most complex market conditions.

These findings significantly validate and extend the foundational work of \textcite{zhang2025calibratinghestonmodeldeep}. While our overall calibration \ac{mre} of 4.33\% appears numerically similar to the 4.64\% reported in the reference study for a 100-option dataset, our result was achieved on a far more complex and diverse calibration surface that was, on average, 2.6 times larger. This suggests that our methodological enhancements, most likely the dynamic calculation of market-implied rates and the use of a quasi-Newton optimizer (L-BFGS-B), have led to superior scalability and robustness under real-world conditions.

The success of the framework can be attributed to the synergistic combination of its core components. The practice of Sobolev training, which penalizes errors in both price and parameter sensitivities, is the key mechanism that enforces the geometric integrity of the pricing surface, enabling the strong generalization observed. This high-fidelity surrogate model, trained with a state-of-the-art `swish` activation function and an AdamW optimizer, is sufficiently precise to guide the final L-BFGS-B optimizer to a high-quality minimum. The final validation with QuantLib's \texttt{AnalyticHestonEngine} confirms that the parameters found are not artifacts of the surrogate model but are genuinely representative of the Heston model's best fit to the market.

From a practical standpoint, this hybrid approach offers a compelling solution to the trade-off between calibration speed and accuracy. The sub-second calibration time makes the methodology suitable for applications requiring frequent recalibration, such as intra-day risk management. The overall true Heston \ac{mre} of 4.90\% should not be viewed as a failure of the calibration, but rather as a realistic measurement of the Heston model's inherent limitations in capturing every feature of the true implied pricing surface. This level of accuracy is highly suitable for applications such as relative value trading, where the goal is to identify significant mispricings against a reliable baseline, and for large-scale risk reporting, where the immediate availability of calibrated Greeks is paramount.

However, the analysis also revealed limitations that suggest avenues for future research. The error analysis in Figure \ref{fig:mre_heatmaps} showed significantly higher relative errors for cheap, out-of-the-money options, a known artifact of the \ac{mre} metric that is amplified by error propagation through the put-call parity formula. Future iterations could address this by training the model directly on the implied volatility surface and implementing a Vega-weighted loss function, which would prioritize the fitting of contracts that are most sensitive to volatility and most relevant to traders and risk managers. Furthermore, while the \ac{ddn} provides a robust point estimate of the Heston parameters, replacing the deterministic network with a Bayesian Neural Network or a deep ensemble could allow the model to provide a credible interval for its predictions. This would offer a quantitative measure of model uncertainty, a feature of significant value for sophisticated risk management applications. Finally, the surrogate modeling approach is particularly promising for models that lack semi-analytical pricing solutions, such as those incorporating jumps or rough volatility, where a network trained on Monte Carlo data could create a fast and accurate pricer for otherwise intractable models. A complementary and highly pragmatic direction for future research is to address the Heston model's residual error with a hybrid, two-stage approach. After the \ac{ddn} calculates the base Heston price, a second "corrector" machine learning model (e.g., a gradient boosting machine) could be trained to predict the systematic pricing error (\(P_{\text{market}} - P_{\text{model}}\)) using option features and the initial price as inputs. The final prediction would then be the sum of the Heston price and this learned correction. This method leverages the \ac{ddn} to enforce the robust, theory-based structure of the Heston model, while the corrector model uses a data-driven approach to learn the model's known systematic biases, such as the volatility smirk. Such a "theory-plus-correction" framework could significantly reduce the final \ac{mre} and push the methodology's accuracy into a range suitable for more demanding pricing applications.
\newpage

\printbibliography[heading=bibintoc]
\newpage

\section*{Use of Artificial Intelligence}
\ac{ai} was employed in the context of this study to support the authors in various auxiliary and editorial tasks. Its use was strictly limited to non-critical and non-confidential aspects of the research, coding and writing process. The primary purpose of employing \ac{ai} tools was to enhance the overall clarity, coherence, and grammatical accuracy of the text, ensuring a consistent academic writing style. Through text proposals and linguistic adjustments, the readability of the thesis was improved, allowing for a clearer communication of complex concepts to the reader. \ac{ai} was additionally utilized for translation purposes between English and German, ensuring linguistic precision in both directions.

Furthermore, \ac{ai}-assisted summarization techniques were employed to help the author better understand complex theoretical or methodological concepts by providing concise explanatory summaries. These summaries served solely as a comprehension aid and were not directly included in the final version of the thesis without the author's verification and adaptation.

In the technical part of the thesis, \ac{ai} was used to generate and refine \texttt{Python} code fragments, as well as to enhance existing code written by the author in terms of readability and computational efficiency. Code completion tools were occasionally used to accelerate programming during the development process. All generated or optimized code was manually integrated into the code base following review and validation by the author. \ac{ai} tools were also employed for the generation of \texttt{LaTeX} table code structures, while the underlying content was entirely created by the author, and for querying specific \texttt{LaTeX} commands to ensure accurate formatting. In addition, \ac{ai} was used to generate concise and descriptive \texttt{Git} commit messages for efficient version control and documentation of code changes.

At no point were personal data, business or trade secrets, or any proprietary data, particularly data downloaded from Bloomberg, processed or shared with any \ac{ai} tool. The handling of such data remained entirely under the author's control and within secure environments compliant with data protection and confidentiality standards.

All textual recommendations and code suggestions provided by \ac{ai} tools were thoroughly reviewed and verified by the author for technical correctness and factual accuracy. This included empirical testing of all code outputs to ensure functionality and to prevent the inclusion of any erroneous or fabricated content, often referred to as hallucinations. The authors assumes full responsibility for the correctness and validity of all content and code presented in this thesis.

\ac{ai} was explicitly not used beyond the purposes listed above. The authors assumes full responsibility for the correctness and validity of all content and code presented in this study. Table \ref{tab:list_of_ai_tools} provides an overview of the specific \ac{ai} models utilized, including their version information and intended use cases within the scope of this study.
% Table of AI Tools Used
\captionsetup{list=no}
\begin{table}[H]
	\centering
	\begin{threeparttable}
		\caption{\textbf{Overview of AI Models Utilized}}
		\label{tab:list_of_ai_tools}
		\small
		\begin{tabular}{p{4cm} p{12cm}}
			\toprule
			\textbf{AI Model}             & \textbf{Citation Information}                                                                                                                                                                                                                                                                                                                                                                                                      \\
			\midrule
			ChatGPT (GPT-5 Model)         &
			OpenAI, \textit{ChatGPT (GPT-5 Model)}, 2025, Version: GPT-5, Used for text refinement including improving grammar, writing style and consistency as well as text proposals. Furthermore, it was used to create \texttt{LaTeX} table structures and or querying specific \texttt{LaTeX} commands to ensure accurate formatting, \url{https://chat.openai.com}                                                                                                        \\

			ChatGPT (GPT-5 Thinking Mini) &
			OpenAI, \textit{ChatGPT (GPT-5 Thinking Mini Model)}, 2025, Version: GPT-5 Thinking Mini, Used for text refinement including improving grammar, writing style and consistency as well as text proposals. Furthermore, it was used to create \texttt{LaTeX} table structures and or querying specific \texttt{LaTeX} commands to ensure accurate formatting. Note this is the fallback model if the GPT-5 model free quota is exceeded, \url{https://chat.openai.com} \\

			Gemini 2.5 Pro                &
			Google DeepMind, \textit{Gemini 2.5 Pro}, 2025, Version: 2.5 Pro, Used for the generation and refinement of code fragments as well as for enhancement of existing code written by the author in terms of readability and computational efficiency. Furthermore it was used to summarize complex theoretical concepts, \url{https://gemini.google.com}                                                                                                              \\

			Windsurf Chat (Llama 3.1 70B) &
			Codeium / Windsurf, \textit{Windsurf Chat Model (based on Meta’s Llama 3.1 70B)}, 2025, Version: Llama 3.1 70B, Used for coding assistance in the IDE as well as for the creation of Git commit messages, \url{https://codeium.com/windsurf}                                                                                                                                                                                                                       \\

			DeepL Translator              &
			DeepL SE, \textit{DeepL Translator}, 2025, Version: 2025 Release, Used for high-quality translation and linguistic consistency checking, \url{https://www.deepl.com}                                                                                                                                                                                                                                                                                               \\
			\bottomrule
		\end{tabular}
	\end{threeparttable}
\end{table}

% Appendix
\newpage
\section{Appendix}
\label{sec:appendix}
\subsection{Visual Analysis of Synthetic Data Distributions}

The visual analysis of the synthetic dataset confirms the successful implementation of the sampling strategy and highlights the complexity of the learning task. The data distributions are presented in Figures \ref{fig:synth_stats_part1} through \ref{fig:synth_stats_part3}.

Figure \ref{fig:synth_stats_part1} displays the histograms and box plots for the primary Heston model parameters: kappa, lambda, sigma, rho, v0, and the risk-free rate r. A defining characteristic of these plots is the perfectly flat, uniform distribution of the histograms and the symmetry of the box plots. This visual evidence validates the use of \ac{lhs}, ensuring that the high-dimensional input space is covered evenly without gaps or clustering. This uniformity is essential for training a neural network that generalizes well across the entire parameter domain.

Figure \ref{fig:synth_stats_part2} continues this analysis with the remaining inputs, tau and log-moneyness, which also exhibit perfect uniformity. However, the subsequent columns in Figure \ref{fig:synth_stats_part2} and the plots in Figure \ref{fig:synth_stats_part3} reveal the distributions of the model outputs: the option price and the parameter sensitivities (gradients). In sharp contrast to the inputs, these output variables display highly non-uniform distributions.

The gradients, particularly d\_kappa (Figure \ref{fig:stats_part2}) and d\_v0 (Figure \ref{fig:synth_stats_part3}), are characterized by extreme leptokurtosis, with sharp peaks around zero and heavy tails containing significant outliers. For instance, the box plot for d\_kappa shows a dense concentration of values near zero but extends to extreme outliers, indicating regions of the parameter space where the option price is highly sensitive to changes in the mean-reversion speed. Similarly, d\_sigma exhibits a negative skew, while d\_lambda is positively skewed. These complex, heavy-tailed distributions underscore the challenge of the regression task, as the \ac{ddn} must learn to map uniformly distributed inputs to highly non-linear and peaked output surfaces.

\begin{figure}[H]
    \centering
    \includegraphics[width=1.0\textwidth]{../data/descriptive_analysis/synthetic_data/combined_statistics_plot_part_1.png}
    \caption{Distributions of the synthetic Heston input parameters (kappa, lambda, sigma, rho, v0, r). The perfectly flat histograms and symmetric box plots confirm the effectiveness of the Latin Hypercube Sampling method in covering the input space uniformly.}
    \label{fig:synth_stats_part1}
\end{figure}

\begin{figure}[H]
    \centering
    \includegraphics[width=1.0\textwidth]{../data/descriptive_analysis/synthetic_data/combined_statistics_plot_part_2.png}
    \caption{Distributions for time to maturity and log-moneyness (inputs), followed by option price and gradients for kappa, lambda, and sigma (outputs). Note the transition from uniform inputs to highly skewed and peaked output distributions.}
    \label{fig:synth_stats_part2}
\end{figure}

\begin{figure}[H]
    \centering
    \includegraphics[width=1.0\textwidth]{../data/descriptive_analysis/synthetic_data/combined_statistics_plot_part_3.png}
    \caption{Distributions for the gradients with respect to rho and v0. These labels exhibit significant positive skewness and heavy tails, indicating the presence of regions with high parameter sensitivity.}
    \label{fig:synth_stats_part3}
\end{figure}

\subsection{Visual Analysis of Historical Data Distributions}
The histograms and box plots presented in this section provide a granular visual analysis of the filtered historical \ac{aapl} options dataset used for the empirical backtest. These visualizations corroborate the descriptive statistics presented in Section \ref{subsec:descriptive_data_analysis} and highlight the non-normal nature of financial market data.

Figure \ref{fig:stats_part1} displays the distributions for the bid and ask prices, underlying asset price, strike price, days to expiration (DTE), and call implied volatility. A dominant feature across the pricing variables (C\_BID, C\_ASK) is the extreme positive skewness. The histograms show a high concentration of option prices near zero, with a long right tail extending to over \$160. The corresponding box plots confirm this via a dense cluster of outliers in the upper range, representing deep in-the-money contracts or options during periods of high volatility.

The distributions for the underlying asset price and strike price are multimodal. This structure reflects the historical price evolution of AAPL stock over the seven-year observation period, where the stock price spent significant time at different valuation levels (e.g., \$150, \$300). The alignment between the underlying and strike distributions confirms that the dataset maintains a consistent moneyness relationship throughout the timeline.

Figure \ref{fig:stats_part2} extends this analysis to put implied volatility (P\_IV), the mid-market call price, log-moneyness, and time to maturity in years (Tau\_Years). The implied volatility distributions for both calls (Figure A.1) and puts (Figure A.2) are highly leptokurtic. They exhibit a sharp peak around the mean volatility level (approximately 30-35\%) and massive right tails with outliers exceeding 800\% (IV > 8.0). These extreme outliers correspond to market stress events, such as the COVID-19 crash, where uncertainty spiked dramatically.

The distribution of time to maturity (DTE and Tau\_Years) is heavily right-skewed, indicating that the dataset is dominated by short-term options. The histogram shows a rapid decay in frequency as maturity increases, which is consistent with the liquidity profile of the equity options market where trading volume is concentrated in the front months.

Finally, the distribution of Log-Moneyness (Moneyness\_Log) in Figure A.2 stands in contrast to the other variables. It displays a relatively symmetric, bounded distribution centered at zero. This is a direct result of the data filtering protocol which restricted the dataset to options with log-moneyness between -0.25 and 0.25. The absence of outliers in the moneyness box plot confirms that the filtering logic was applied correctly, ensuring that the calibration focused strictly on the liquid, near-the-money region of the volatility surface.

\begin{figure}[H]
    \centering
    \includegraphics[width=1.0\textwidth]{../data/descriptive_analysis/historic_aapl_data/combined_statistics_plot_part_1.png}
    \caption{Histograms and box plots for Call Bid, Call Ask, Underlying Price, Strike Price, DTE, and Call Implied Volatility. The plots reveal significant right-skewness in pricing and volatility variables, and a multimodal distribution for the underlying asset.}
    \label{fig:stats_part1}
\end{figure}

\begin{figure}[H]
    \centering
    \includegraphics[width=1.0\textwidth]{../data/descriptive_analysis/historic_aapl_data/combined_statistics_plot_part_2.png}
    \caption{Histograms and box plots for Put Implied Volatility, Call Mid-Price, Log-Moneyness, and Time to Maturity (Years). Note the heavy tails in the volatility distribution and the symmetric, bounded nature of the log-moneyness resulting from data filtering.}
    \label{fig:stats_part2}
\end{figure}

\subsection{Longitudinal Analysis of Calibrated Parameter Stability}

Figure \ref{fig:parameters} illustrates the daily evolution of the five calibrated Heston parameters ($\kappa, \lambda, \sigma, \rho, v_0$) over the full backtesting period from 2016 to 2023. This time-series analysis provides critical insights into the stability of the calibration and the model's response to changing market regimes.

The most financially intuitive behavior is observed in the initial variance parameter ($v_0$), displayed in the bottom panel. It acts as a robust proxy for market fear, exhibiting low, mean-reverting behavior during calm periods (2016-2017) and sharp, distinct spikes during stress events. The most prominent spike corresponds to the COVID-19 crash in March 2020, where $v_0$ surged to approximately 0.7, correctly reflecting the explosion in spot volatility. Smaller spikes are visible during the volatility event of early 2018 and the market correction of 2022.

The long-run variance ($\lambda$) remains remarkably stable for the majority of the period, hovering between 0.05 and 0.10. This indicates that despite short-term fluctuations in spot variance, the model's view of the long-term equilibrium volatility remained anchored. A notable exception is the singular, extreme spike in early 2018, likely associated with the "Volmageddon" event, where the sudden collapse of short-volatility strategies momentarily disjointed the long-term expectation.

The correlation parameter ($\rho$) generally adheres to the empirical leverage effect, staying in negative territory between -0.4 and -0.8. However, a regime shift is observable during the 2020-2021 recovery period. During this phase, $\rho$ frequently hits the upper boundary of 0.0, and the volatility of volatility ($\sigma$) simultaneously hits its upper boundary of 1.0. This boundary-hitting behavior suggests that the standard Heston model struggled to accommodate the specific smile dynamics of that period---likely characterized by steep skews and high prices for far-out-of-the-money calls---without pushing its parameters to the limits of the constrained search space.

Finally, the mean reversion speed ($\kappa$) exhibits high variance and noise throughout the entire sample, oscillating rapidly between 1.0 and 4.0. This is a well-documented phenomenon in Heston calibration, where $\kappa$ often acts as a slack parameter, absorbing residual fitting errors that the other parameters cannot capture.

\begin{figure}[H]
    \centering
    \includegraphics[width=1.0\textwidth]{../plots/plot_6_parameters.png}
    \caption{Daily calibrated Heston parameters over the backtesting period (2016-2023). The plots reveal both intuitive market responses (e.g., spikes in $v_0$ during crises) and challenges in model fitting (e.g., boundary-hitting behavior of $\rho$ and $\sigma$ during 2020-2021).}
    \label{fig:parameters}
\end{figure}

\subsection{Data Source Files for Historical AAPL Options}
The historical \ac{aapl} options data utilized in this study is sourced from two comprehensive CSV files obtained from Kaggle. These files encompass daily option quotes spanning from January 2016 to December 2023. The files can be downloaded from the following link: https://www.kaggle.com/datasets/kylegraupe/aapl-options-data-2016-2020
\newpage

\end{document}